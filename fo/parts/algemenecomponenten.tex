
\section{Algemene componenten}
\label{sec:algemenecomponenten}

\subsection{Logo en naam}
\label{sec:logoennaam}
Op een beheerpagina in Drupal kan de naam van de website en het logo aangepast worden. Dit gaat door middel van een tekstveld in te vullen en een afbeelding te uploaden via een knop die ene bestand op de computer selecteert. Door op het logo of naam te klikken gaat de bezoeker naar de homepagina.

\subsection{Meta menu}
\label{sec:metamenu}
Het meta menu is een menu met links naar pagina's, dit menu heeft standaard de volgende items:
\begin{itemize}
  \item Home
  \item Sitemap
  \item Adressen en openingstijden
  \item Veelgestelde vragen
  \item RSS (Bij het klikken van deze link wordt de RSS getoond)
  \item Lees voor (Bij het klikken van deze link wordt Readspeaker geopend)
\end{itemize}
Dit menu is redactioneel uit te breiden.

\subsection{Tabs}
\label{sec:tabspagina}
De tabs is optioneel, deze functionaliteit kan in- en uitgeschakeld worden via een beheerpagina in Drupal door middel van een checkbox. %todo verder uitwerken

\subsection{Hoofdnavigatie}
\label{sec:hoofdnavigatie}
Het eerste niveau van het menu wordt getoond. Met een mouseover of een focus door middel van de tab-toets op het toetsenbord klapt het submenu van het betreffende hoofditem naar onderen open. Hier wordt onder elkaar het tweede niveau getoond. Bij het klikken of op de enter toets drukken ga je naar het betreffende pagina die aan het menu hangt.

\subsection{Zoekveld}
\label{sec:zoekveld}
Bij het zoekveld kan er gezocht worden door de website naar content op de website en content in documenten. Na het invullen van het zoekwoord kan er geklikt worden op de zoekbutton of op de entertoetst gedrukt worden, als de tekstcursor in het zoekveld staat, waarna de bezoeker naar het zoekresultatenpagina wordt geleidt. 

\subsection{Alfabetische balk}
\label{sec:alfabetischebalk}
De alfabetische balk is optioneel, deze functionaliteit kan in- en uitgeschakeld worden via een beheerpagina in Drupal door middel van een checkbox. Alle letters van het alfabet worden getoond, ook diegene die geen resultaat hebben. De functionaliteit verschilt hier; Javascript aan of Javascript uit in de browser.

\subsubsection{Javascript aan}
Bij het klikken op een letter verschijnt onder de balk een blok met daarin links naar pagina's die beginnen met de letter waarop gedrukt is. Als het blok is uitgeklapt en de bezoeker klik op een andere letter dan klapt het huidige blok weer in en verschijnt een andere blok met resultaten van de letter waarop is geklikt. De inhoud hiervan is redactioneel aan te passen. Er wordt geen uitzondering van gedrag gemaakt als de bezoeker twee keer op dezelfde letter klikt. Bij het klikken op het kruisje klapt het blok in.

\subsubsection{Javascript uit}
Bij het klikken op een letter wordt de bezoeker naar een pagina geleidt met redactioneel te bepalen content die begint met de letter waar de bezoeker op heeft geklikt.
