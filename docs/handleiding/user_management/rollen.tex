\subsection{Rollen}\label{rollen}

Rollen zijn een synoniem voor \emph{gebruikers typen}, bijvoorbeeld: \emph{Anonieme gebruiker} en \emph{Administrator}.  Het is mogelijk om rollen specifieke rechten te geven. Op deze manier kun je bijvoorbeeld bepaalde gebieden of functionaliteiten van de website verbergen of juist tonen. 

De onderstaande tabel toont een overzicht van geactiveerde functionaliteiten per rol.
\emph{J} staat voor \emph{Ja} en \emph{N} staat voor \emph{Nee}. 

\begin{tabularx}{\textwidth}{ | p{5cm} |X|X|X|X| }
  \hline
  Actie & Beheerder & Eindredacteur & Redacteur & Medewerker \\ \hline
  Webformulieren  & J  & J  & N & N  \\ \hline
  Felix beheer  & J  & N  & N & N  \\ \hline
  Menu beheer  & J  & J  & N & N  \\ \hline
  Subsite beheer  & J  & J  & N & N  \\ \hline
  Taxonomie beheer  & J  & J  & N & N  \\ \hline
  Gebruikersbeheer  & J  & J  & N & N  \\ \hline
\end{tabularx}


\subsubsection{Administrator}\label{administrator}
Gebruikers met de rol \emph{Administrator} hebben ongelimiteerd toegang tot de gehele website. De \emph{Administrator} kan alle content bekijken, rechten toekennen en bijvoorbeeld menu's aanmaken en bewerken.

\subsubsection{Beheerder}\label{beheerder}
Gebruikers met de rol \emph{Beheerder} hebben gelimiteerde toegang tot de website. 

\textbf{Belangrijkste taken Beheerder:}

\begin{enumerate}
\item Contentmanagement\seeone{contentmanagement}
\item Gebruikersbeheer\seeone{usermanagement}
\item Algemeen Menubeheer\seeone{menu}
\item Alfabet menu\seeone{alfabet}
\end{enumerate}

\subsubsection{Medewerker}\label{medewerker}
Gebruikers met de rol \emph{Medewerker} hebben een gelimiteerde toegang tot de website. Medewerkers kunnen o.a. content publiceren, nieuwe content toevoegen op de interne marktplaats en nieuwe wiki pagina's aanmaken.

\textbf{Belangrijkste taken Medewerker:}

\begin{enumerate}
\item Content toevoegen\seeone{contenttoevoegen}
\item Foto\seeone{foto} en marktplaats\seeone{marktplaats}  items toevoegen
\end{enumerate}

\subsubsection{Redacteur}\label{redacteur}
Gebruikers met de rol \emph{Redacteur} hebben een gelimiteerde toegang tot de website. Redacteuren hebben o.a. toegang tot het beheermenu, mogen contextuele links gebruiken en kunnen inhoud aanmaken van alle inhoudstypen. Redacteuren mogen geen content publiceren.

\textbf{Belangrijkste taken Redacteur:}

\begin{enumerate}
\item Nieuws \seeone{nieuws}
\item Eenvoudige pagina\seeone{eenvoudigepagina}
\item WYSIWYG \seeone{wysiwyg}
\item Agenda \seeone{agenda}
\item Taxonomie \seeone{taxonomie}
\end{enumerate}

\subsubsection{Eindredacteur}\label{eindredacteur}
Gebruikers met de rol \emph{Eindredacteur} hebben een ongelimiteerde toegang op het gebied van content. Eindredacteuren mogen content aanmaken, wijzigen, publiceren en verwijderen. Ook hebben eindredacteuren toegang tot het beheermenu en mogen zij de contextuele links gebruiken.

\textbf{Belangrijkste taken Eindredacteur:}

\begin{enumerate}
\item Workflow\seeone{workbench}
\item Content publiceren\seeone{contentbeheer}
\item Nieuwsbrief\seeone{nieuwsbrief}
\item Webformulieren\seeone{webform}
\end{enumerate}

\subsubsection{Teamlid}\label{teamlid}
De rol \emph{Teamlid} is bedoeld voor subsites. Deze rol kan aan een bestaand persoon toegekend worden om haar toegang te geven tot content op subsites en om haar toestemming te geven om content aan te maken op de opgegeven subsite(s).

\textbf{Belangrijkste taken Teamlid:}

\begin{enumerate}
\item Content toevoegen\seeone{contenttoevoegen}
\item Foto\seeone{foto} en marktplaats\seeone{marktplaats}  items toevoegen
\end{enumerate}