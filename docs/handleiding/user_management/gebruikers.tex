\subsection{Geregistreerde gebruikers}\label{geregistreerdegebruikers}

Geregistreerde gebruikers kunnen bepaalde acties uitvoeren op de website waar anonieme gebruikers dat niet kunnen. 
De onderstaande afbeelding toont de \emph{Gebruikers interface} met al haar opties en mogelijkheden.

\bigskip

\begin{center}
	\includegraphics[width=\textwidth]{img/gebruikers1.png}
\end{center}

\subsubsection{Gebruikers toevoegen}

Gebruikers kunnen aangemaakt worden via de frontend(registratie) en via de backend. Deze paragraaf beschrijft hoe je manueel gebruikers kunt toevoegen via de backend. 

Ga naar \emph{Personen} en klik vervolgens op \emph{Gebruiker toevoegen}, of ga direct naar \drupalpath{admin/people/create}.

Vul alle verplichte velden in en selecteer eventueel welke rol de gebruiker dient te hebben.
Klik onderaan de pagina op \emph{Nieuw account aanmaken} om de gebruiker toe te voegen.

\subsubsection{Gebruikers bewerken}

Ga naar \emph{Personen} en klik bij de betreffende gebruiker in de meest rechter kolom op \emph{bewerken} om het account te gaan bewerken. 
Bewerk het account naar wens en klik vervolgens onderdaan de pagina op de knop \emph{Opslaan} om de wijzigingen op te slaan.

\subsubsection{Gebruikers verwijderen}

Ongewenste gebruikers accounts kunnen verwijderd worden uit de database; de volgende manier is de meest makkelijke:
Ga naar \emph{Personen} en kruis in de meest linker kolom aan welke gebruiker je wilt gaan verwijderen, het is mogelijk om meerdere gebruikers tegelijk te verwijderen. Selecteer bij \emph{Update-instellingen} welke actie je wilt gaan uitvoeren, klik in dit geval op \emph{Geselecteerde gebruikersaccounts annuleren}. Om de gebruiker(s) daadwerkelijk te verwijderen uit de database klik je op de knop \emph{Bijwerken}. Let op: deze actie is permanent en kan niet (zomaar) ongedaan worden gemaakt. 