\subsection{Inhoudstypen}\label{inhoudstypen}
Inhoudstypen worden gebruikt voor het weergeven van verschillende typen inhoud. Per inhoudstype kunnen bijvoorbeeld verschillende velden gedefinieerd worden en de bijbehorende pagina's kunnen apart gestijled worden. Het gebruik van inhoudstypen cre�ert veel flexibiliteit.

In de volgende paragraaf zullen specifieke inhoudstypen nader beschreven worden.

\subsubsection{Agenda}\label{agenda}
Agenda

\subsubsection{Bekendmaking}\label{bekendmaking}
Bekendmaking

\subsubsection{Bericht}\label{bericht}
Bericht

\subsubsection{Bestand}\label{bestand}
Bestand

\subsubsection{Bestemmingsplan}\label{bestemmingsplan}
Bestemmingsplan

\subsubsection{Blog}\label{blog}
Blog

\subsubsection{Editorial}\label{editorial}
Editorial

\subsubsection{Eenvoudige pagina}\label{eenvoudigepagina}
Eenvoudige pagina

\subsubsection{FAQ}\label{faq}
FAQ

\subsubsection{Forumonderwerp}\label{forumonderwerp}
Forumonderwerp

\subsubsection{Foto}\label{foto}
Foto

\subsubsection{Marktplaats}\label{marktplaats}
Marktplaats

\subsubsection{Nieuws}\label{nieuws}
Nieuws

\subsubsection{Onderwerp}\label{onderwerp}
Onderwerp

\subsubsection{Peiling}\label{peiling}
Peiling

\subsubsection{Persoon}\label{persoon}
Persoon

\subsubsection{Product}\label{product}
Product

\subsubsection{RSS}\label{rss}
RSS

\subsubsection{RSS Source}\label{rsssource}
RSS Source

\subsubsection{Regeling}\label{regeling}
Regeling

\subsubsection{Slide}\label{slide}
Slide

\subsubsection{VAC}\label{vac}
Vac

\subsubsection{Wiki}\label{wiki}
Het inhoudstype 'wiki' zal worden gebruikt voor het aanmaken van informatieve pagina's.

Om een wiki pagina aan te maken ga je naar: \emph{Inhoud toevoegen} $\rightarrow$ \emph{Wiki} , of ga direct naar \drupalpath{node/add/wiki}.

Vul een titel bij het veld 'Titel' en vul informatieve tekst in bij het veld 'Body'.

Voor het inhoudstype 'Wiki' is een speciale filter gebouwd, deze filter maakt het refereren naar andere wiki pagina's mogelijk. 

Het refereren naar een andere wiki pagina gaat als volgt: 

\begin{enumerate}
\item Kopieer de titel van de node waarnaar je wilt refereren, bijvoorbeeld 'wiki pagina'
\item In de body tekst van je huidige node kun je refereren naar de node 'wiki pagina'
\item Plaats de node waarnaar je wilt refereren tussen dubbele 'square brackets': [[wiki pagina]]
\item Bijvoorbeeld: dit is informatie, op de pagina [[wiki pagina]] kun je meer informatie vinden.
\end{enumerate}

\bigskip

\begin{center}
	\includegraphics[width=\textwidth]{img/wiki.png}
\end{center}