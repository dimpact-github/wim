\subsection{Sitemap}\label{sitemap}
De sitemap wordt standaard ingesteld tijdens de installatie van Dimpact. Mocht je de sitemap toch naar wens willen aanpassen dan kun je dit doen via de volgende link: \drupalpath{admin/config/search/sitemap}

In dit formulier kun je aangeven wat er in de sitemap moet komen te staan. Voer bij \emph{Paginatitel} een titel voor de pagina in. Onder \emph{Sitemap bericht} kan je een tekst plaatsen die op de pagina zichtbaar is. 

Onder \emph{SITEMAP INHOUD} kan je kiezen voor de volgende instellingen. \emph{Geef de voorpagina weer}, plaats een link naar de voorpagina. \emph{Menu's die in de sitemap moeten worden opgenomen}, geef aan welke menu's getoond moeten worden als lijstweergave op de sitemap. \emph{Geef FAQ inhoud weer}, toont een lijst met FAQ items. \emph{Categorieen die in de sitemap moeten worden opgenomen}, toont een lijst met taxonomietermen die binnen de vocabulaires vallen.

Onder \emph{CATEGORIE-INSTELLINGEN} stel je de weergavemodus in voor de te tonen taxonomietermen. \emph{Show node counts by categories} toont het aantal gekoppelde nodes aan een taxonomieterm. \emph{Categories depth} geef de diepte op tot op welk niveau termen opgehaald moeten worden. \emph{thresholds} geven aan vanaf hoeveel gekoppelde items de term weergegeven moet worden.

Onder \emph{RSS-INSTELLINGEN} stel je de instellingen in voor de RSS feed. Onder \emph{CSS SETTINGS} kan je opgeven of je het CSS bestand bij de RSS feed wilt inladen.

Klik onderaan de pagina op de knop \emph{Instellingen opslaan} om te instellingen op te slaan. De sitemap wordt dan beschikbaar op /sitemap.