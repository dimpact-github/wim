\subsection{Headerafbeelding}
\pvelist{ \pve{2.5} }

Met behulp van de headerimage module kan een node(slide) getoond worden
boven: alle pagina's, een set van pagina's of een specifieke pagina.

Het is mogelijk om een headerimage te overschrijven, bijvoorbeeld:
'headerimage 1' staat ingesteld op '/', dit betekent dat 'headerimage 1' op alle
pagina's getoond zal worden. Stel je 'headerimage 2' in op 'contact', dan zal 'headerimage 2' boven de contact pagina getoond worden en 'headerimage 1' op alle andere pagina's.

\subsubsection{Headerafbeelding toevoegen}
Ga naar \drupalpath{admin/config/prorail/headerimage/add}.

Vul bij 'URL Pad' een pad in, bijvoorbeeld: 'contact'.
Vul bij 'Node titel' de titel in van de node die weergegeven moet worden in de header.

Klik op de knop �Opslaan� om de headerafbeelding toe te voegen.

\subsubsection{Headerafbeelding wijzigen}
Klik op �Deze headerimage wijzigen� en volg dezelfde stappen als bij �Headerimage toevoegen�

\subsubsection{Headerafbeelding overschrijven}
Klik op �Deze headerimage overschrijven� en volg dezelfde stappen als bij �Headerimage toevoegen�.

\subsubsection{Headerafbeelding wijzigen}
Klik op �Deze headerimage verwijderen� en bevestig. Let op: deze actie kan niet teruggedraaid worden.

\begin{figure}[p]
\centering
\includegraphics[width=\textwidth]{img/headerafbeelding.png}
\end{figure}