\subsection{Contentdeling}\label{contentdeling}
\pvelist{ \pve{2.13}, \pve{2.13.1}, \pve{2.13.2}, \pve{2.13.3}, \pve{2.13.4} }
Het is mogelijk om inhoud op meerdere subsites te gebruiken. Een nieuwsartikel kan bijvoorbeeld zichtbaar zijn op de subsites van zowel reizigers als omwonenden.

Bij het toevoegen of bewerken van een node staat er onderin het formulier een item \emph{Contentdeling}. Na het klikken op de link \emph{Contentdeling} wordt een lijst getoond met alle subsites. Vink de subsites aan waarop de inhoud beschikbaar moet zijn. Het is verplicht om minimaal \'{e}\'{e}n subsite aan te vinken. Onder de lijst van domeinen kan een brondomein worden gekozen. Dit domein (subsite) wordt gebruikt als \emph{canonical url}. Dit is de URL die door zoekmachines zal worden gebruikt om de pagina te indexeren. Tevens wordt deze subsite gebruikt wanneer naar dit item wordt gelinkt vanuit een andere subsite waar dit item niet is gepubliceerd. Bij het aanmaken van nieuwe inhoud is standaard het huidige domein als brondomein geselecteerd.
