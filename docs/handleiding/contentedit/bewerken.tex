\subsection{Content bewerken}

\pvelist{ \pve{2.2.1}, \pve{2.2.4} }

\begin{enumerate}
\item Ga naar \emph{Mijn Workbench} $\Rightarrow$ \emph{Inhoud zoeken} of ga direct naar \drupalpath{admin/workbench/all-content}, je zult nu de pagina te zien krijgen waar alle inhoud te vinden is.
\item Om de gewenste inhoud sneller te vinden kun je de filterfunctie gebruiken, bovenaan de pagina. Klik bijvoorbeeld bij het label "type" in het lijstje op "Standaard content item" en vervolgens klik je op "Toepassen". Alleen artikelen van het type "Standaard content item" zullen nu getoond worden.
\item Zoek naar de titel van het artikel dat je wilt gaan bewerken en klik vervolgens op "bewerken" in de meest rechtse kolom genaamd "handelingen".
\item Je bent nu gereed om de content te gaan bewerken. Om de bewerkingen op te slaan klik je onderaan de pagina op de knop "Opslaan".
\end{enumerate}

De bestaande pagina blijft tijdens de bewerkmodus beschikbaar voor bezoekers.

Het is niet mogelijk om met meerdere redacteuren aan dezelfde pagina te werken. Het bewerkscherm is toegankelijk, maar bij het opslaan zal de volgende melding verschijnen: \emph{"De inhoud op deze pagina is gewijzigd door een andere gebruiker, of u heeft al wijzigingen ingediend via dit formulier. De wijzigingen kunnen daardoor niet worden opgeslagen."}.

\subsubsection{Hoe plaats je een afbeelding naast een stuk tekst}
Dit kun je doen door de afbeelding toe te voegen in de lopende tekst. Rechtermuisknop op de afbeelding en kies dan voor eigenschappen.Onderaan kun je kiezen voor de uitlijning.
%todo: https://dutchopen.unfuddle.com/a#/projects/106/tickets/by_number/253 screenshots