
\subsection{Downloads}
\pvelist{ \pve{2.16.1}, \pve{2.16.2}, \pve{2.16.4}, \pve{2.16.5}, \pve{2.16.6} }
Downloads kunnen via een redactioneel blok worden toegevoegd. Ga hiervoor naar \emph{Inhoud toevoegen} $\rightarrow$ \emph{Redactioneel blok} of direct naar \drupalpath{node/add/redactioneel-blok}. Klik onder \emph{Downloads} op de knop \emph{Media selecteren}. Klik vervolgens op de knop \emph{Bladeren}. De browser zal hier standaard de laatst gebruikte locatie openen. Selecteer het gewenste bestand en klik daarna op de knop \emph{Uploaden}. Klik op de knop \emph{Doorgaan}. Hierna kan eventueel metadata worden ingevuld.

Na het toevoegen van een download is het mogelijk om nog een download toe te voegen. Als er meerdere downloads zijn toegevoegd dan worden deze onder elkaar in hetzelfde blok getoond. Op deze manier kan een documenteniljst op een pagina worden geplaatst.

Naast het toevoegen van nieuwe bestanden kan ook een eerder ge\"{u}pload bestand worden gekozen. Klik hiervoor na \emph{Media selecteren} op het tabblad \emph{Bibliotheek}. Deze lijst is standaard gesorteerd op de uploaddatum, waardoor het eerste bestand het bestand is wat als laatste is toegevoegd.

Na het toevoegen van een redactioneel blok met downloads moet dit blok worden toegevoegd aan een pagina. Zie hiervoor het onderdeel over flexibele blokken\seeone{felix}.

\subsubsection{Originele afbeelding downloaden}
\begin{enumerate}
\item Ga naar \drupalpath{admin/workbench/files}
\item Filter op bestandstype Afbeelding
\item Klik in de filelist met de rechtermuisknop op een plaatje en kies Eigenschappen
\item Kopieer de link en haal dan het stuk \emph{styles/588x304/public/} eruit
\end{enumerate}

