
\subsection{Taxonomie}\label{taxonomie}

Taxonomie wordt gebruikt voor lijsten van woorden. Bijvoorbeeld: de lijst �FAQ categorie�n� bevat de woorden \emph{Netwerk} en \emph{Werkplekken}. Taxonomie lijsten kunnen altijd aangepast worden.

\bigskip

\begin{center}
	\includegraphics[width=\textwidth]{img/taxonomie2.png}
\end{center}

\subsubsection{Woorden aan woordenlijsten toevoegen}

Ga naar het overzicht van alle woordenlijsten. Klik op \emph{Termen toevoegen} bij de lijst waaraan je woorden wil toevoegen. Vul bij het veld \emph{Naam} het woord in en vul eventueel een beschrijving in. Klik onderaan de pagina op de knop \emph{Opslaan} om het woord aan de lijst toe te voegen. Herhaal deze stappen om meerdere woorden toe te voegen aan de woordenlijst. 

\subsubsection{Woorden in woordenlijsten bewerken}

Het is mogelijk om specifieke woorden uit lijsten te bewerken. Klik op \emph{Termen weergeven} bij de betreffende lijst.
Klik op \emph{Bewerken} bij het betreffende woord, vervolgens kun je de wijzigingen doorvoeren. Klik op de knop \emph{Opslaan} om de wijzigingen op te slaan.

\subsubsection{Woorden uit woordenlijsten verwijderen}

Het is mogelijk om specifieke woorden uit lijsten te verwijderen. Klik op \emph{Termen weergeven} bij de betreffende lijst.
Klik op \emph{Bewerken} bij het betreffende woord, om het woord te verwijderen klik je onderaan de op de knop \emph{Verwijderen}. Na het bevestigen zal het woord definitief en onherstelbaar verwijderd worden.
