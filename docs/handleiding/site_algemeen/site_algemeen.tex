
\section{Site algemeen}\label{sitealgemeen}
In dit hoofdstuk worden de overige functionaliteiten van de website uitgelegd.

\subsection{Responsive Design}\label{responsivedesign}

\drupalpath is op gezet met een Responsive Design. Dit houdt in dat de website zijn uiterlijk aanpast naar de resolutie van verschillende platformen (bijvoorbeeld een mobile telefoon of tablet). 

Andere platformen hebben vaak kleinere browser vensters. Het design van Dimpact zal hier rekening mee houden zodat de website functioneel en overzichtelijk blijft. Dit kun je ook in je eigen browser testen door de breedte van de browser handmatig te veranderen.
\subsection{Favorieten}\label{favorieten}

Via de \usemodule{views} en \usemodule{flag} module wordt het voor medewerkers mogelijk om pagina's als "favoriet" aan te merken. De lijst van favorieten is vervolgens terug te vinden op het dashboard\seeone{dashboard}. Er wordt een nieuw flag type aangemaakt met de volgende settings:
\begin{itemize}
\item Title: "Favoriet" (machine name: "favorite")
\item Flag link text: "Toevoegen als favoriet"
\item Unflag link text: "Verwijderen uit favorieten"
\item Flag access: medewerker- en redacteurrollen
\item Bundles: agenda, basic page, blog, wiki
\item Display link as field: aanvinken
\end{itemize}
Voor alle overige settigs wordt de standaardinstelling gebruikt. Met de laatste setting wordt een link toegevoegd aan de detailpagina's die in de favorieten kunnen worden gezet.



\subsection{Taxonomie}\label{taxonomie}

Taxonomie wordt gebruikt voor lijsten van woorden. Bijvoorbeeld: de lijst �FAQ categorie�n� bevat de woorden 'Netwerk' en 'Werkplekken'. Taxonomie lijsten kunnen altijd aangepast worden.

\subsubsection{Woordenlijsten toevoegen}

Om nieuwe woordenlijsten toe te voegen ga je naar: \emph{Structuur} $\rightarrow$ \emph{Taxonomie} $\rightarrow$ \emph{Woordenlijst toevoegen}, of ga direct naar \drupalpath{admin/structure/taxonomy/add}.

Vul een naam in bij het veld 'Naam' en geef eventueel een beschrijving op bij het veld 'Beschrijving'.
Klik op de knop 'Opslaan' om de woorden lijst toe te voegen.

Om woorden toe te voegen aan een lijst klik je in de meest rechter kolom op 'Termen toevoegen' bij de betreffende lijst.
Vul bij het veld 'Naam' het woord in dat je aan het lijstje wilt toevoegen, bijvoorbeeld 'Netwerk'. 
Klik onderaan de pagina op de knop 'Opslaan' om het woord toe te voegen aan de lijst. 
Herhaal deze stap om meerdere woorden toe te voegen.

\bigskip

\begin{center}
	\includegraphics[width=\textwidth]{img/taxonomie1.png}
\end{center}


\subsubsection{Woordenlijsten bewerken}

Om de naam en/of beschrijving van de woordenlijst te bewerken klik je op 'Woordenlijst bewerken' bij de betreffende lijst.
Na het wijzigen klik je op de knop 'Opslaan' om de wijzigingen op te slaan.

Het is ook mogelijk om specifieke woorden uit lijsten te bewerken. Klik op 'Termen weergeven' bij de betreffende lijst.
Klik op 'Bewerken' bij het betreffende woord, vervolgens kun je de wijzigingen doorvoeren. Klik op de knop 'Opslaan' om de wijzigingen op te slaan.

\bigskip

\begin{center}
	\includegraphics[width=\textwidth]{img/taxonomie2.png}
\end{center}

\subsubsection{Woordenlijsten verwijderen}

Een woordenlijst kan in ��n keer verwijderd worden, klik op 'Woordenlijst bewerken' bij de betreffende lijst en klik vervolgens op de knop 'Verwijderen'. Na het bevestigen zal de lijst definitief en onherstelbaar verwijderd worden.

Het is ook mogelijk om specifieke woorden uit lijsten te verwijderen. Klik op 'Termen weergeven' bij de betreffende lijst.
Klik op 'Bewerken' bij het betreffende woord, om het woord te verwijderen klik je onderaan de op de knop 'Verwijderen'. Na het bevestigen zal het woord definitief en onherstelbaar verwijderd worden.

\subsection{Zoeken}\label{zoeken}
Voor de zoekfunctionaliteit maken we gebruik van de \usemodule{apachesolr} module. We gaan uit van default settings (dat wil zeggen: configuratiebestanden die meegeleverd worden bij de \texttt{apachesolr} module). Berekeningen en settings van relevantie, synoniemenlijsten etc. zullen we ongemoeid laten. Het is vaak ook niet noodzakelijk om dit aan te passen aangezien de default settings doorgaans goede resultaten leveren.

DOP levert de configuratiebestanden voor Solr aan.

\subsubsection{Zoekbox in header}
Een zoekopdracht invoeren gaat altijd via de zoekbox in de header. Voor dit blok wordt de standaard \usemodule{search} module uit Drupal core gebruikt.

\subsubsection{Zoekresultaten}
De zoekresultaten worden getoond titel en snippet. De zoekwoorden in de snippet tonen we met gele markering (aanpasbaar via CSS).

\paragraph{Zoekbox}
Op de zoekresultatenpagina staat de zoekbox zowel in de header als in de linkerkolom. Om het blok dubbel te kunnen gebruiken maken we gebruik van de \usemodule{block\_clone} module. Via de \texttt{bespoke} module zorgen we met \emph{preprocess hooks} ervoor dat het blok niet in de content wordt getoond en dat de zoekopdracht in de zoekbox blijft staan.

\paragraph{Facetten}
Links van de zoekresultaten wordt een blok toegevoegd met de mogelijkheid om verder te filteren. Hiervoor wordt de \usemodule{facetapi} module gebruikt. De volgende filtering wordt gebruikt:
\begin{itemize}
\item Nodetype
\item Publicatiedatum (per maand)
\item Doelgroep (taxonomie)
\end{itemize}

\subsubsection{Zoeken in bijlagen}
Bijlagen die zijn toegevoegd aan nodes worden meegenomen bij het indexeren. Hiervoor wordt \emph{Apache Tika} i.c.m. de \usemodule{apachesolr\_attachments} module gebruikt. Tika kan diverse formaten omzetten naar platte tekst die geschikt is voor indexatie. Hieronder vallen ook PDF en Microsoft Office formaten\footnote{http://tika.apache.org/1.4/formats.html}.

\subsubsection{Bias settings}
De \usemodule{apachesolr} module biedt standaard Bias settings aan voor velden en types. Deze settings zijn beschikbaar voor de admin user. De mogelijkheden beschrijven we in de handleiding.

\subsubsection{Spellingscontrole}
De optie \emph{spellingscontrole} wordt ingeschakeld (onder de settings van de zoekpagina). De woordenlijst komt standaard uit de ge\"{i}ndexeerde tekst. Later kan - indien wenselijk - een eigen woordenlijst worden toegevoegd. Die zal echter generiek zijn voor alle Dimpact gemeenten en kan niet via de admin worden toegevoegd.

\subsubsection{Zoeken binnen landelijke voorzieningen}
De ge\"{i}mporteerde data (inclusief landelijke voorzieningen, RIS en PDC) wordt meegenomen met de zoekresultaten. 


\subsection{Afbeeldingsstijlen}\label{afbeeldingsstijlen}

Afbeeldingsstijlen worden gebruikt voor het weergeven van afbeeldingen. Verschillende afbeeldingsstijlen maken het mogelijk een afbeelding op een andere manier (ander formaat bijvoorbeeld) te tonen dan het origineel. 

\subsubsection{Banner}

\textbf{Banner specificaties:} Schalen en bijsnijden 363x125 pixels 


\subsubsection{Full width}

\textbf{Full width specificaties:} Schalen en bijsnijden 1120x200 pixels 


\subsubsection{List thumbnail}

\textbf{List thumbnail specificaties:} Schalen en bijsnijden 100x100 pixels 


\subsubsection{Node thumbnail}

\textbf{Node thumbnail specificaties:} Schalen en bijsnijden 532x225 pixels  


\subsubsection{Origineel}

\textbf{Origineel specificaties:} geen bewerkingen 


\subsubsection{Overige onderwerp}

\textbf{Overige onderwerp specificaties:} Schalen en bijsnijden 390x100 pixels  


\subsubsection{Portret}

\textbf{Portret specificaties:} Schalen en bijsnijden 120x120 pixels  


\subsubsection{Slide 9/3}

\textbf{Slide specificaties:} Schalen en bijsnijden 740x200 pixels 


\subsubsection{galleryformatterslide}

\textbf{Galleryformatterslide specificaties:} Schalen en bijsnijden 500x312 pixels 


\subsubsection{galleryformatterthumb}

\textbf{Galleryformatterthumb specificaties:} Schalen en bijsnijden 121x75 pixels  


\subsubsection{Thumbnail}

\textbf{Thumbnail specificaties:} Schalen 100x100 pixels (vergroten toegestaan)


\subsubsection{Medium}

\textbf{Medium specificaties:} Schalen 220x220 pixels (vergroten toegestaan) 


\subsubsection{Large}

\textbf{Large specificaties:} Schalen 480x480 pixels 


\subsubsection{linkitthumb}

\textbf{Linkitthumb specificaties:} Schalen 50x50 pixels 


\subsubsection{squarethumbnail}

\textbf{Squarethumbnail specificaties:} geen bewerkingen 

\subsection{Abonnementen}\label{abonnementen}
Onderaan detailpagina's van types waarop geabonneerd mag worden staat een formulier om je te abonneren op deze pagina, alle pagina's van dit type of alle pagina's van dit type van deze gebruiker. Selecteer een of meerdere opties en druk op Opslaan.

\begin{center}
	\includegraphics[scale=0.75]{img/abonnementen.png}
\end{center}

Het beheren van de abonnementen staat beschreven in het hoofdstuk \emph{Abonnementen}\seeone{profileabonnementen}.

\subsection{Storingspagina}

Er is een storingspagina aanwezig die getoond wordt bij technische problemen. De tekst van deze pagina is niet redactioneel aan te passen en wordt ingesteld bij installatie.

\subsection{Reageren}

Op de meeste contenttypes kan worden gereageerd. Content types die zijn uitgesloten van reactiemogelijkheden zijn rss, rss\_source (intranet nieuws), slide en editorial (redacitonele blokken).

\subsection{Geocoding}\label{geocoding}

Voor de Google Maps functionaliteit is het mogelijk om automatisch de lengtegraad en breedtegraad op te halen bij de ingevoerde adressen. Hiervoor is een API key noodzakelijk welke aangevraagd kan worden bij Google. De API key kan door een beheerder in Drupal worden ingesteld.
\begin{enumerate}
\item Ga naar \texttt{https://code.google.com/apis/console/} en login met een Google-account
\item Klik in de linkerkolom op \emph{Services}
\item Zet \emph{Google Maps Geolocation API} aan
\item Klik links op \emph{API Access}
\item Kopieer de tekst achter "API key:"
\item Ga naar \texttt{admin/config/content/location/geocoding}
\item Kies voor \emph{Google Maps} bij \emph{Nederland}
\item Sla de settings op
\item Klik op \emph{Parameters instellen} achter \emph{Nederland}
\item Vul de API key in
\item Klik op \emph{Instellingen opslaan}
\end{enumerate}
