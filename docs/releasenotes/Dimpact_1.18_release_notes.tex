% latex packages to load.
\documentclass[12pt]{article}
\usepackage{geometry}
\geometry{a4paper}
%\geometry{landscape}
\usepackage[parfill]{parskip}    % Activate to begin paragraphs with an empty line rather than an indent
\usepackage{graphicx}
\usepackage{amssymb}
\usepackage{epstopdf}
\usepackage{fancyhdr}
\usepackage{fullpage}
\usepackage{appendix}
\usepackage{newclude}
\usepackage{datetime}
\usepackage{hyperref}
\usepackage{color}
\usepackage{multicol}
\usepackage{tabularx}
\usepackage{enumerate}
\usepackage{enumitem}
\usepackage{listings}
\usepackage{varwidth}
\usepackage{wallpaper}
\usepackage{lastpage}
\usepackage{titling}
%\usepackage{multind}
\usepackage[dutch]{babel}
\usepackage[table]{xcolor}
\usepackage{mathtools}
\usepackage{amsmath}
\usepackage{MnSymbol}
\usepackage{wasysym}

\usepackage[default,osfigures,scale=0.90]{opensans} %% Alternatively
%% use the option 'defaultsans' instead of 'default' to replace the
%% sans serif font only.
\usepackage[T1]{fontenc}

%tikz images, unquote to use block, color block, autograph block, cloud, line and line node
\usepackage[framemethod=tikz]{mdframed}
\usepackage{tikz}
\usetikzlibrary{shapes, arrows}

\tikzstyle{block} = [rectangle, draw, text width = 6em, text centered, rounded corners, minimum height = 4em]
\tikzstyle{colorblock} = [rectangle, draw, fill = blue!20, text width = 6em, text centered, rounded corners, minimum height = 4em]
%\tikzstyle{inputblock} = [rectangle, draw, text width = 24em, minimum height = 2.5em]
%\tikzstyle{autographblock} = [rectangle, draw, fill = black!1, text height = 0, text depth = 2cm, text width = 24em, minimum height = 8em]
\tikzstyle{cloud} = [ellipse, draw, minimum height = 4em]
\tikzstyle{line} = [draw, -latex']
\tikzstyle{line node} = [draw, fill = white]
\tikzstyle{decision} = [diamond, aspect=2, draw, text badly centered]

\usepackage[official]{eurosym}

\newcommand{\usemodule}[1]{\index{modules}{#1}\texttt{#1}}

\definecolor{gray}{rgb}{0.5,0.5,0.5}
\definecolor{tableheader}{rgb}{0.7,0.7,0.7}

\setlist[description]{style=nextline}
\renewcommand{\familydefault}{\sfdefault}

% link setup
\hypersetup{
    colorlinks,
    citecolor=black,
    filecolor=black,
    linkcolor=black,
    urlcolor=black,
}

\DeclareGraphicsRule{.tif}{png}{.png}{`convert #1 `dirname #1`/`basename #1 .tif`.png}

% usefull commands:
\newcommand{\seeref}[1]{\ref{#1} p.\pageref{#1}}
%\newcommand{\see}[1]{ (zie \ref{#1} p.\pageref{#1})}
\newcommand{\seesee}[2]{ (zie \ref{#1} p.\pageref{#1},  \ref{#2} p.\pageref{#2})}

% style for code blocks
\lstset{
    linewidth=1\textwidth,
    breaklines=true,
    numbers=left,                   % where to put the line-numbers
    numberstyle=\tiny\color{gray},  % the style that is used for the line-numbers
    stepnumber=1,                   % the step between two line-numbers. If it's 1, each line
    numbersep=5pt,
    basicstyle=\footnotesize,
}

% Header and Footer settings
\URCornerWallPaper{0.13}{img/dop/gglogo.png}
\pagestyle{fancy}
\fancyhead{}
\renewcommand{\headrulewidth}{0pt}


\fancyfoot[L]{Release notes}
\fancyfoot[C]{  }
\fancyfoot[R]{\textbf{\thepage}\ / \pageref{LastPage} \linebreak \linebreak \_\_\_\_\_\_\_ }

% Settings for table of contents.
\setcounter{secnumdepth}{4}
\setcounter{tocdepth}{3}


\makeatletter
% some extra spacing for the table of contents
\renewcommand{\l@subsection}{\@dottedtocline{2}{1.5em}{3em}}
\renewcommand{\l@subsubsection}{\@dottedtocline{2}{2.7em}{4em}}

\renewcommand\paragraph{%
   \@startsection{paragraph}{4}{0mm}%
      {-\baselineskip}%
      {.5\baselineskip}%
      {\normalfont\normalsize\bfseries}}
\makeatother

% Define variables
\newcommand{\customer}{Dimpact release 1.18}
\newcommand{\projectname}{Release notes}
\newcommand{\customerdomain}{Dimpact}
\newcommand{\authors}{R. Ragas}

\title{\textbf{\customer} \\ \projectname}
\pretitle{\begin{flushleft}\LARGE}
\posttitle{\par\end{flushleft}}


\author{}  % skippen we voor maketitle
\date{}

% The actual Document:
\begin{document}
\maketitle
\vspace{-2.6cm}
\begin{flushright}
\begin{tabularx}{4.8cm}{ X }
GoalGorilla     \\
Oudemarkt 9B         \\
7511 GA Enschede         \\
T: 0800-GOAL    \\
\\*
\\*
\\*
\\*
\\*
\\*
\\*
\\*
\\*
\\*
\\*
\\*
\\*
\\*
\\*
\\*
\\*
\\*
\footnotesize
\footnotesize
\end{tabularx}
\end{flushright}

 \null
 \vfill
  \begin{tabularx}{\linewidth}{ p{4cm} X }
    Plaats & Enschede                \\
    Laatst bijgewerkt & \ddmmyyyydate \today    \\
    Auteur & \authors              \\
    Versie & concept                \\
  \end{tabularx}
\pagebreak


% table of contents

\clearpage

%\renewcommand*\contentsname{Inhoudsopgave}
%\tableofcontents
%\pagebreak

\section{Wijzigingen}
\subsection{Security wijzigingen}

Onderstaande wijzigingen zijn gerealiseerd ten behoeve van de exploitatie van de Dimpact-websites.

\begin{tabular}{| r | p{15cm} |}
  \hline \# & Omschrijving \\ \hline
  GS-1941 & Module update
  \begin{itemize}
    \item Drupal core (Drupal 7.50 $\rightarrow$ 7.54)
    \item Workbench Moderation (workbench\_moderation 7.x-1.x-dev $\rightarrow$ 7.x-3.x)
    \item Elysia Cron (elysia\_cron 2.1 $\rightarrow$ 2.4)
    \item Mailchimp (mailchimp 2.13 $\rightarrow$ 4.8)
    \item Webform (webform 3.24 $\rightarrow$ 3.25)
    \item Google Analytics (google\_analytics 2.1 $\rightarrow$ 2.3)
    \item Hotjar (hotjar 1.0 $\rightarrow$ 1.2)
    \item Metatag (metatag 1.0 $\rightarrow$ 1.21)
    \item Views (views 3.14 $\rightarrow$ 3.15)
  \end{itemize} \\ \hline
  GS-1941 & Core

  Deze module is de basis van een drupal website. Hierop wordt de website gebouwd met alle features die erbij horen. \\ \hline
  GS-1941 & Workbench moderation

  Deze module wordt gebruikt om content te modereren die via de workbench is aangemaakt. \\ \hline
  GS-1941 & Elysia Cron

  Deze module wordt gebruikt om cronjobs met meer mogelijkheden te draaien. Cronjobs draaien op de achtergrond om ervoor te zorgen dat systeemtaken worden uitgevoerd. \\ \hline
  GS-1941 & Mailchimp

  Deze module wordt gebruik om mailchimp te implementeren binnen de website. Dit geeft de mogelijkheid tot inschrijven op nieuwsbrieven. \\ \hline
  GS-1941 & Webform

  Deze module wordt gebruikt om webformulieren aan te maken. \\ \hline
  GS-1941 & Google Analytics

  Deze module wordt gebruikt om inzicht te geven in hoe bezoekers de website gebruiken. Dit door het gedrag te tracken. \\ \hline

  GS-1941 & Hotjar

  Deze module wordt net als Google analytics gebruikt om inzicht te geven in hoe bezoekers de website gebruiken. Dit door het gedrag te tracken. \\ \hline

  GS-1941 & Metatag

  Deze module wordt gebruikt om metatags in te voeren voor content welke weergegeven worden in zoekmachines. \\ \hline

  GS-1941 & Views

  Deze module wordt gebruikt om overzichten te maken met specifieke content. Denk hierbij aan bijvoorbeeld een nieuwsoverzicht \\ \hline
\end{tabular}

\section{Testresultaten}
  De gelevere versie heeft een aantal punten waar op gelet moet worden.
  \begin{itemize}
    \item Elysia Cron - De crontab is veranderd van elysia-cron $\rightarrow$ elysia-cron run
    \item Mailchimp - De mailchimp update vereiste een library upgrade. Hierdoor is het nodig om de ingestelde mailchimp gegevens na te kijken via instellingen $\rightarrow$ webservices $\rightarrow$ mailchimp.
  \end{itemize}
\end{document}
