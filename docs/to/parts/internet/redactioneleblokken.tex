\subsection{Redactionele blokken}\label{felix}

De module \usemodule{felix} wordt ingezet om het voor redacteuren mogelijk te maken om redactionele blokken te plaatsen binnen vooraf gedefinieerde regio's. Welke regio's dat zijn wordt in deze sectie verder uitgewerkt. Alle blokken die worden geplaatst zijn specifiek voor \'{e}\'{e}n pagina en worden dus niet automatisch op meerdere (gerelateerde) pagina's geplaatst. De blokken zijn wel generiek over alle subsites. Wanneer een blok op \texttt{/nieuws} wordt geplaatst dan zal deze op een andere subsite ook zichtbaar zijn op dat pad, mits de geplaatste node ook op dat domein is gepubliceerd.

Onderstaand tabel geeft een overzicht van de Felix regio's.

\begin{tabularx}{\linewidth}{| p{5cm} | p{3cm} | X |}
\hline
\rowcolor{tableheader}
\textbf{Naam} & \textbf{Systeemnaam} & \textbf{Differentiate content per} \\ \hline
Linkerkolom & left & path \\ \hline
Footer kolom 1 & footer1 & domain \\ \hline
Footer kolom 2 & footer2 & domain \\ \hline
Footer kolom 3 & footer3 & domain \\ \hline
\end{tabularx}

Om de blokken te kunnen scheiden per domein wordt gebruik gemaakt van de \usemodule{felix\_domain} submodule.

Er wordt \'{e}\'{e}n \emph{blockset} aangemaakt. De blokken die vrij te plaatsen zijn kunnen dus in elke regio worden gezet. De theming is wel afhankelijk van de regio (wordt volledig in CSS geregeld). In de rest van dit onderdeel wordt aangegeven welke blokken er mogelijk zijn, met screenshots.

