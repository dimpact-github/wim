
\subsection{Hoofdnavigatie}

De hoofdnavigatie is zichtbaar in de header (alleen hoogste niveau).

\subsection{Topmenu}

Voor het topmenu wordt een apart menu aangemaakt. Hiervoor wordt het standaard menublok uit Drupal core gebruikt.

\subsection{Alfabet}

Voor het alfabet wordt een apart menu gemaakt. Hierin kunnen onder de letters sub-items worden aangemaakt met snelkoppelingen naar de betreffende pagina's. Deze worden gebruikt voor de overlay. Redactioneel is er dus volledige vrijheid welke links hier worden geplaatst. Pagina's achter de letters zelf worden gemaakt via \usemodule{views}. Hiervoor wordt een view aangemaakt op het pad \texttt{letter/\%}, waarbij de \% een contextual filter is op nodetitel ("begint met"). De links gaan dus naar \texttt{letter/a}, \texttt{letter/b} etc. Voor oplevering worden alle	 26 items (a t/m z) aangemaakt.

\subsection{Subnavigatie}

De subnavigatie komt in de linkerkolom op pagina's waar dat van toepassing is. Dit blok wordt gebouwd met de \usemodule{submenutree} module. Dit menu begint op het derde niveau en kan t/m het zesde niveau tonen (dus bevat 3 lagen).

\subsection{Footer}

De footer bestaat uit 3 kolommen en is vrij in te vullen door de redactie. Hiervoor worden 3 Felix regio's aangemaakt\seeone{felix}. De inhoud van de footer is voor elke pagina van de subsite gelijk.

