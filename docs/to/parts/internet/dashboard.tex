\subsection{Dashboard voor redacteuren}\label{workbenchdashboard}

Voor de redactie wordt een dashboard ontwikkeld op basis van de \usemodule{views} en \usemodule{views\_bulk\_operations} modules. Het dashboard komt op \texttt{/admin/workbench}. Een deel van deze pagina's komt uit de \texttt{workbench} module. De rest wordt via \usemodule{views} en \usemodule{emptypage} toegevoegd. De landingspagina krijgt hiermee ruimte voor een aantal widgets, te weten:
\begin{itemize}
\item Mijn bewerkingen \\ Lijst van laatste bewerkingen gedaan door de huidige gebruiker
\item Alle recente inhoud \\ Lijst van items die recent zijn toegevoegd
\end{itemize}
De volgende tabbladeren worden toegevoegd:
\begin{itemize}
\item Inhoud aanmaken \\ Kopie van de \texttt{node/add}. Deze komt op \texttt{admin/workbench/create} en komt standaard uit \texttt{workbench}.
\item File list \\ View met alle bestanden (toegevoegd via de \usemodule{media} / \usemodule{file\_entity} module).
\item Reacties \\ View met alle reacties
\item Inhoud zoeken \\ View met alle nodes en filtermogelijkheid
\end{itemize}
Bij het zoeken van inhoud zijn de volgende filters beschikbaar:
\begin{itemize}
\item Gepubliceerd (ja / nee)
\item Alle woorden
\item Een van de woorden
\item Bevat niet
\item Subsite (dropdown)
\item Type (dropdown)
\item Tag (autocomplete)
\item Auteur (autocomplete)
\item Items per pagina (dropdown)
\end{itemize}
Tevens zijn er bulk opties beschikbaar om meerdere items tegelijk te verwijderen of te (de)publiceren.
