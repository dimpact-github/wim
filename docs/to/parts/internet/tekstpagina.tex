\subsection{Tekstpagina}\label{tekstpagina}

Voor de tekstpagina's wordt het standaard nodetype \texttt{page} gebruikt.

Het is mogelijk om in de tekst afbeeldingen op te nemen via de media bibliotheek\seeone{media}. Onder de bodytekst kunnen blokken worden toegevoegd voor bijv. een carrousel\seeone{felix}.

\subsubsection{Inline tabs}

Op tekstpagina's moet het mogelijk zijn om inline tabs op te nemen. In plaats van \'{e}\'{e}n stuk tekst wordt de indeling dan als volgt:
\begin{verbatim}
[introtekst]
[tabs]
[tekst van actieve tab]
\end{verbatim}
De actieve tab is altijd de eerste, totdat een ander tabblad wordt aangeklikt. Het aantal tabs is ongelimiteerd. Wel kan er maar \'{e}\'{e}n set tabs per pagina worden gebruikt.
De invoer van de tabs gebeurt in het body veld. Elke heading die begint met "**" wordt omgezet naar een tab. De volgende invoer:
\begin{verbatim}
<p>Introtekst</p>
<h2>Titel</h2>
<p>Meer tekst...</p>
<h2>**Eerste</h2>
<p>Tekst in eerste tabblad</p>
<h2>**Tweede</h2>
<p>Tekst in tweede tabblad</p>
<p>Meer tekst in tweede tabblad</p>
\end{verbatim}
wordt omgezet naar:
\begin{verbatim}
<p>Introtekst</p>
<h2>Titel</h2>
<p>Meer tekst...</p>
<div class="inlinetabs">
  <ul>
    <li><a href="#inlinetabs-[id]"><span>Eerste</span></a></li>
    <li><a href="#inlinetabs-[id]"><span>Tweede</span></a></li>
  </ul>
  <div id="inlinetabs-[id]">
    <p>Tekst in eerste tabblad</p>
  </div>
  <div id="inlinetabs-[id]">
    <p>Tekst in tweede tabblad</p>
    <p>Meer tekst in tweede tabblad</p>
  </div>
</div>
\end{verbatim}
De tekst \texttt{[id]} is een unieke code voor dat tabblad en wordt random gegenereerd. De omzetting vind plaats in een input filter en wordt ge\"{i}mplementeerd in een custom module met de naam \texttt{inlinetabs}. Hierin wordt ook een JavaScript toegevoegd om de tabs actief te maken middels jQuery UI Tabs\footnote{http://api.jqueryui.com/tabs/}.

Het nieuwe input filter wordt aangezet op \emph{filtered html} en \emph{full html}. Dit filter wordt aan het eind toegevoegd om te voorkomen dat de id's en classes verdwijnen.

\subsubsection{404 en 403 pagina's}\label{404pagina}

De 404 en 403 pagina's zijn een speciaal soort tekstpagina.

Om te voorkomen dat deze pagina's in de zoekresultaten komen wordt een aanpassing in de \texttt{bespoke} module gedaan zodat deze pagina's niet meegenomen wordt in het zoekresultaat (via de \texttt{apachesolr\_query\_alter}-hook).

Deze pagina is niet gelijk aan de storingspagina voor fouten uit de \texttt{5xx}-categorie\seeone{storingspagina}.

