\subsection{Wysiwyg en media}\label{wysiwyg}

Voor media en wysiwyg zullen we de \usemodule{wysiwyg} en \usemodule{media} / \usemodule{file\_entity} modules inzetten. Als editor zelf wordt gekozen voor \texttt{CKEditor} aangezien dit de meestgebruikte editor is binnen Drupal en standaard is in Drupal 8.

\subsubsection{Buttons}

We zullen de volgende buttons instellen:
\begin{itemize}
\item Bold
\item Italic
\item Linkit
\item Media
\end{itemize}

\subsubsection{Links in bodytekst}

Voor het gemakkelijk kunnen toevoegen van links in bodyteksten zullen we gebruik maken van de \usemodule{linkit} module i.c.m. \usemodule{pathologic}.

\subsubsection{Invoerformaten}

Er worden twee invoerformaten ter beschikking gesteld voor gebruikers en redacteuren.

\begin{itemize}
\item \textbf{Filtered HTML} \\
Standaard filtering op ongewenste HTML en herschrijven van links door \texttt{pathologic}.
\item \textbf{Plain text} \\
Wordt gebruikt voor reviews (geplaatst door bezoekers). Geen opmaak / HTML toegestaan.
\end{itemize}

\subsubsection{Media}\label{media}

Images en video's moeten toegevoegd en hergebruikt kunnen worden in wysiwyg en bestandsvelden.
We gaan uit van versie 2 van de \texttt{media} module.


