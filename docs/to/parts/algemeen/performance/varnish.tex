\subsection{Varnish}\label{varnish}
\textit{Varnish} is een \textit{reverse proxy server}. Dit is een server tussen de client en webserver in die dient als extra caching laag.
\textit{Varnish} is in staat om pagina's van niet-ingelogde bezoekers te cachen. Deze caching bied uitstekende bescherming tegen het \textit{slashdot-effect} \footnote{http://en.wikipedia.org/wiki/Slashdot\_effect}.

Om het nadelige effect van caching (het blijven zien van oude content) te verhelpen wordt de \usemodule{purge} module gebruikt om pagina's uit de \textit{Varnish}-cache te halen wanneer deze worden aangepast.

\subsubsection{Varnish configuratie}

\paragraph{vcl\_recv}\label{varnishrecv}

De purge module maakt een call naar de Varnish server met ``PURGE`` als requesttype. Om deze in Varnish af te handelen zetten we het volgende bovenaan in de \textit{vcl\_fetch}:

\begin{lstlisting}[language=C]
  if (req.request == "PURGE" || req.request == "PURGEALL") {
    if (!client.ip ~ purge) {
      // Return error code 405 (Forbidden) when not
      error 405 "Not allowed.";
    }
    if (req.request == "PURGE") {
      // Purge all objects from cache that match the incoming url and host
      ban("req.url == " + req.url + " && req.http.host == " + req.http.host);
    }
    else {
      // Purge all objects from cache that match the incoming host
      ban("req.url ~ ^/ && req.http.host == " + req.http.host);
    }
    // Return a http error code 200 (Ok)
    error 200 "Purged";
  }
\end{lstlisting}

We normaliseren de \texttt{Accept-Encoding} header om dezelfde cache te kunnen hanteren voor de diverse varianten die de browsers meesturen.

\begin{lstlisting}[language=C]
  if (req.http.Accept-Encoding) {
    if (req.url ~ "\.(jpg|png|gif|gz|tgz|bz2|tbz|mp3|ogg|pdf)$") {
      // No point in compressing these
      remove req.http.Accept-Encoding;
    } elsif (req.http.Accept-Encoding ~ "gzip") {
      set req.http.Accept-Encoding = "gzip";
    } elsif (req.http.Accept-Encoding ~ "deflate") {
      set req.http.Accept-Encoding = "deflate";
    } else {
      // unkown algorithm
      remove req.http.Accept-Encoding;
    }
  }
\end{lstlisting}

We gebruiken tevens dezelfde cache voor elke browser:

\begin{lstlisting}[language=C]
  set req.http.user-agent = "Mozilla";
\end{lstlisting}

We laten alleen de session, ssl en NO\_CACHE cookies door. Alle overige cookies hebben daarmee geen invloed op de caching.

\begin{lstlisting}[language=C]
  if (req.http.Cookie) {
    set req.http.Cookie = ";" + req.http.Cookie;
    set req.http.Cookie = regsuball(req.http.Cookie, "; +", ";");
    set req.http.Cookie = regsuball(req.http.Cookie, ";(NO_CACHE|ssl|.?SESS[^=]+)=", "; \1=");
    set req.http.Cookie = regsuball(req.http.Cookie, ";[^ ][^;]*", "");
    set req.http.Cookie = regsuball(req.http.Cookie, "^[; ]+|[; ]+$", "");
    if (req.http.Cookie == "") {
      remove req.http.Cookie;
    }
    else {
      // Lookup, the Cookies will be added to the hash.
    }
  }
\end{lstlisting}

\paragraph{vcl\_fetch}

We configureren de \textit{grace period} op 15m.

\begin{lstlisting}[language=C]
  set beresp.grace = 15m;
\end{lstlisting}

Statische bestanden zullen we altijd cachen. PHP-files nooit.

\begin{lstlisting}[language=C]
  if (req.url ~ "\.(png|gif|jpg|txt)$") {
    set beresp.ttl = 1h;
    return(deliver);
  }

  if (req.url ~ "\.(css|js)$") {
    set beresp.ttl = 15m;
    return(deliver);
  }

  if (req.url ~ "\.php$") {
    set beresp.ttl = 20m;
    return(hit_for_pass);
  }
\end{lstlisting}

%\subsubsection{Mobile detection}\label{varnishmobile}

%Voor mobiel wordt een andere pagina gebruikt als frontpage\seeone{mobile}. Detectie daarvoor moet gebeuren aan de hand van de \emph{user agent} string. Om alle caches te kunnen blijven gebruiken zal deze detectie in Varnish moeten worden ingebouwd. Hiervoor is reeds een implementatie beschikbaar\footnote{https://github.com/varnish/varnish-devicedetect/}. Deze implementatie stuurt een nieuwe header mee aan Apache. Dit is \'{e}\'{e}n van de volgende:
%\begin{verbatim}
%X-UA-Device: pc
%X-UA-Device: tablet-ipad
%X-UA-Device: mobile-iphone
%X-UA-Device: tablet-android
%X-UA-Device: mobile-firefoxos
%X-UA-Device: mobile-smartphone
%X-UA-Device: mobile-generic
%\end{verbatim}
%De mobiele variant zullen we gebruiken voor elk apparaat waarbij deze waarde begint met \texttt{mobile}. Voor tablets gebruiken we dus de standaard homepage. In de Varnish configuratie wordt hiervoor de volgende regel bovenaan toegevoegd:

%\begin{lstlisting}[language=C]
%include "devicedetect.vcl";
%\end{lstlisting}

%Binnen de \texttt{vcl\_recv} functie wordt de volgende code toegevoegd (bovenaan de functie):

%\begin{lstlisting}[language=C]
%call devicedetect;
%if (req.http.X-UA-Device ~ "^mobile") {
%  set req.url = "/mobile";
%}
%\end{lstlisting}

%In de \texttt{vcl\_hash} functie zullen we de volgende code toevoegen:

%\begin{lstlisting}[language=C]
%if (req.http.X-UA-Device ~ "^mobile") {
%  hash_data("mobile");
%}
%\end{lstlisting}

%Door deze aanpassing zal de pagina op \texttt{/voorpagina-mobiel} worden geladen bij het opvragen van de frontpage door een mobile device.
