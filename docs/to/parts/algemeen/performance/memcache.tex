\subsection{Memcache}\label{memcache}

Standaard gebruikt Drupal de database voor de opslag van interne caches. De \usemodule{memcache} module is echter een sneller alternatief voor caching. Dit is een stuk software dat cache in het geheugen van de server bewaard. Caching via memcache is (over TCP/IP) meer dan 2 keer zo snel als in de database\footnote{http://www.mysqlperformanceblog.com/2006/08/09/cache-performance-comparison/}. In de praktijk zal dit meer uitmaken wanneer beide op een andere server draaien.
